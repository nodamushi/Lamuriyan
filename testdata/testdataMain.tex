\newenvironment{document}{\tagname{contents}\setAttr{id}{document}}{}
\newenvironment{head}{\tagname{headcontent}\disableparagraph}{}
\let\nTeX\Lamuriyan

\newcommand{\cred}{*f65}


\begin{head}
%  \charset{utf-8}
%  \block{title}{Lamuriyan仮本部}
  \block{meta}{\setAttr{content}{noarchive,noindex,nofollow}\setAttr{name}{robots}}
  \js{http://cdn.mathjax.org/mathjax/latest/MathJax.js?config=MML_HTMLorMML}
  \css{ntex.css}

\end{head}

\setgprop{title}{Lamuriyan仮本部}

\begin{document}
%\begin{div}
%\setAttr{id}{mainbody}
\begin{abstract}
  \label{abstract}
  
    ここは、とりあえず生成したHTMLファイルを置く場所がなかったので忍者ツールズの一時的なアカウントで作ったページです。全く、こういうときはてなダイアリーは困るね(-_-) そのうち、私のHPをつくったらそちらに移動させます。\\[2em]




  さて、本ページは\nTeXのデモページです。

  自分のHPやブログの記事を\TeXで管理したいな、と思ったことはありませんか?\TeXだったら数式を簡単にかけるのにな。\TeXだったら図1とかの番号を特に考えなくても勝手に管理してくれるのにな。でも、今ある\TeXからHTMLを生成するツールで、思いのままにスタイリッシュなサイトを生成するのは力不足感が否めない。たとえば、\verb+\ref+でページ内リンクを張りつつ、chapter単位で別ページに切り出したい、という場合、生成されたHTMLを元手に分離作業をするのでは修正が大変ですね。HTMLになるまえのDOMオブジェクトの状態で結果を出力して欲しい。そういうnodamushiの要望に応える為に\Lamuriyanは作られました。

  \nTeXとは、\TeXライクな文法で、ファイル変換するとHTMLのDOMを生成するJavaで作成されたプログラムです。GitHubで公開しています。\anchor{https://github.com/nodamushi/nTeXforHTML}{ nodamushi / nTeXforHTML}\\[3em]

  \nTeXは\TeXライクな文法なので、当然マクロが使えます。
\begin{code}{tex}
\def\foolish #1様は超絶かっこいい{#1は悲しいほど馬鹿}
\foolish nodamushi様は超絶かっこいい
\end{code}
これを実際に\nTeXソースに書くと↓の様になります

\def\foolish #1様は超絶かっこいい{#1は悲しいほど馬鹿}
\foolish nodamushi様は超絶かっこいい\\[2em]

そして、もちろん数式も使えます。
  \begin{eqnarray}
    \label{eq:eq1}
  \delta(x)&=&\left\{\begin{array}{cl}
      \infty & (\text{if } x=0)\\
      0 & (\text{if } x\neq 0)
    \end{array}\right.\\
  f(x)&=&\frac{1}{\sqrt{2\pi\sigma^2}}\sum_{i=0}^{10}a_i \int_0^x\exp\left(-\frac{(t-\mu)^2}{2\sigma^2}\right)dt
\end{eqnarray}

\\[2em]

ソースコードを以下の様にシンタックスハイライトすることが出来たりします。(現状Java,JavaScript,CoffeeScript,HTML,\TeXに対応。私が使う文にはこれだけで十分なので…)
\begin{code}{javascript}
function Person(name){
  this.name=name;
}
Person.prototype.getName=function(){
  return this.name;
}
var p = new Person("nodamushi");
\end{code}


あなたも、\Lamuriyanを用いてページを生成してみませんか?


\end{abstract}


\documentnav[0-2]

\chapter{Lamuriyanについて}
\label{sec:aboutlamuriyan}
\chapternav

\section{作り始めた動機}
\label{sec:motivation}
  私がいい加減ブログの記事を自分で読む時に探すのが大変なので、HPにまとめようかとしていたときに、やたら私の記事は数式が多いので\TeXでホームページ管理できないかなぁと考え、探してみましたが、
+Javaで使える
+DOM構造をはき出す
+relをリンクに変換する
+マクロが使える

ぐらいの条件で数秒探してみましたが、残念なことに私は探すことが出来ませんでした。

ある女王は言ったそうです。\textbf{\color{*5c9}パンがないなら作ればいいじゃない。}

私もその言葉に従うことにしました。ないなら作ればいいと。そして気がついたのです。\\[4em]

{\large あ、私、\TeXで引数とるマクロすら書いたことなかったわ}

\TeX仕様どころか、引数とるマクロすら書いたことがないのに処理環境を作り始めた世紀の愚者nodamushiによる\TeXライクな何か\Lamuriyanはこうして始まったのです。



\section{\Lamuriyanの方針}
\LamuriyanはHTMLのDOM構造を出力することを最終目標としており、{\color{red}デザインをすることやサイトを構築することを目的としていません。}

\Lamuriyanで出力された結果を別に作ってあるテンプレートに挿入してHTML全体を作る、例えば{\color{*f65}記事の内容を作るような場面}を想定して作っています。もちろん、やろうと思えば\Lamuriyanだけでデザインもすることは出来ますが、わざわざ\Lamuriyanで行うメリットが見当たりません。

デザインはCSSが担当すべきであり、HTMLが担当するべき事ではありません。そのため、\Lamuriyanはなるべくデザインに関する内容は{\color{red}classを出力することで表現します。}たとえば、太文字にしろという\verb+\bftext+などは、<span style="font-weight:bold">を出力するのでは無く、<span class="boldweight">という結果を出力します。実際に太文字にするかどうか、どういうデザインにするかはCSSに一任します。なので、\Lamuriyan上では太文字にしろ、という命令でも、実際には細字にしても構いませんし、テキストシャドウを用いて表現しても構いません。

サイト構築に関しても、\Lamuriyanの立ち位置はページを出力、もしくはページの一部を生成するツールという立ち位置を意識しており、サイト管理、構成は\Lamuriyanを用いるプログラムに任せることを前提としています。\\[3em]

\ref[,章]{sec:motivation}でも述べたように、処理系を作り始めるにはお粗末な\TeXの知識しか無く、\textbf{\color{*f46}\TeXとの完全な互換を目的とした物では\udecorationありません。}

あくまで、\TeXライクな\textbf{何か}であり、過去の資産の再利用は気にしていません。過去に書いた\TeXの記事を軽い修正でHTMLに出来ればな、ぐらいの心づもりです。

過去の資産の利用ではなく、これから記事を書いていくときに、\textbf{\color{*36f}\TeX形式の数式を使うことが出来}、\textbf{\color{*36f}数式などの番号管理をする必要が無く}、\textbf{\color{*36f}<section>タグとかも書かなくても自動で生成され}、\textbf{\color{*36f}同じ所はマクロで自動化することが出来}、あと私が過去に作った\textbf{\color{*36f}ソースコードをシンタックスハイライトするプログラムを自動で呼び出すことが出来る}、そんなシステムを目指しています。

その他の精神は以下のような物です。
+動けばいい
+実行速度は求めない
+省メモリは求めない
+本家\TeXと挙動が違う!バグだ!?否、仕様である

\vspace{2em}

\Lamuriyanを単純にMathMLを出力するツールとしてだけ使いたいという要望が万一ありましたら、DOMを生成するのでは無く、直接文字列を出力する為の簡易ツールを公開する………かもしれません。

\section{\Lamuriyanの由来}
\Lamuriyanは{\bf\color{red} n}odamushiが作る\TeX環境ということで、nTeXと呼んでいましたが、ググってみたら普通にいっぱい出てきました。というわけでnTeXから改名して\Lamuriyanという名前に変えました。この仮本部はnTeXの名前で作ったので、URLがntexです。

名前の由来は「LaTeX目指してみたけど、マジ無理やん」を略して「{\large La無理やん}」です。漢字ひらがなローマ字混じりだと打つのが面倒くさいので{\Lamuriyan}となりました。\Lamuriyanでは\verb+\Lamuriyan+とかくことで表示されます。頭文字のLAはLaTeXから来てるので、AがLの上に乗ります。\Lamuriyanと複雑な表記をしたくないときはLamuriyanと書いてください。






\chapter{文章を書く}
基本的な文法は\TeXに乗っ取っていますが、HTMLで表示する事を目的とした結果の違いがいくつかあります。

\chapternav


\section{改行をする}
\TeXでは改行をしても、生成される出力結果は改行されません。\Lamuriyanもソースコードで改行しても、結果には反映されません。改行をするには{\color{red}\verb+\\+}を用います。実際に文章中で使うと\\
このように改行されます。ただし、段落が生成されるわけではないので、行頭の字下げが起こっていないことが確認できるかと思います。{\color{green}(字下げは\Lamuriyanの機能では無く、CSSで指定しています。)}

\verb+\\+は\verb+\\[3em]+の様に記述することで、改行後の行送りの幅を指定することが出来ます。\\[3em]
こんなかんじにね。


\section{段落を生成する}
\Lamuriyanでは二回以上の改行があると新しい段落<p>を生成します。

\begin{code}{tex}
ここは第一段落です。
一回の改行は段落も改行も生成しません。改行したいときは\\を用いてください。

ここは第二段落です。
\end{code}

\vspace{2em}

<p>を発行して欲しくないときには、{\color{red}\verb+\disableparagraph+}コマンドを使うと、それ以降の同じグループでは段落<p>が生成されなくなります。再度段落を作りたい場合は{\color{red}\verb+\enableparagraph+}を利用します。これらの設定は同一グループの中だけで有効です。現在、段落を生成するかどうかは、\verb+\useparagraph+コマンドで確認することが出来ます。この値がfなら段落は生成されず、tなら、段落が生成されます。なお、\verb+\useparagraph+の値を変更しても、段落を生成するかどうかのフラグは変化しないので、確認する以外には用いないでください。

なお、段落と段落の間の空白を作りたいときは\verb+\vspace+を利用してください。



\section{フォントの大きさを変える}
フォントの大きさを変えるには、\verb+\tiny,\scriptsize,\footnotesize,\large,\Large,\LARGE,\huge\Huge+を用います。

これらはそれぞれ<span class="~~size">を出力します。class名は以下を参考にして、実際のフォントサイズはCSSを用いて指定してください。

-\verb+\tiny+:class=tinysize  一般に50\%のサイズのフォントです。
-\verb+\scriptsize+:class=scriptsize   一般に70\%のサイズのフォントです。
-\verb+\footnotesize+:class=footnotesize  一般に80\%のサイズのフォントです。
-\verb+\large+:class=largesize  一般に120\%のサイズのフォントです。
-\verb+\Large+:class=xlargesize  一般に140\%のサイズのフォントです。
-\verb+\LARGE+:class=xxlargesize  一般に170\%のサイズのフォントです。
-\verb+\huge+:class=hugesize  一般に210\%のサイズのフォントです。
-\verb+\Huge+:class=xhugesize  一般に250\%のサイズのフォントです。
-\verb+\normalsize+:デフォルトフォントサイズです。

\subsection{フォントサイズを追加する}
\label{sec:addfontsize}
上記だけでおよそ十分だと思いますが、よりフォントサイズを追加したい場合もあるかもしれません。

\Lamuriyan処理系は\verb+\fontsize+というコマンドの値からフォントの大きさを定義する{\color{\cred}class名}を読み取っています。この値が{\color{\cred}"nomal"}である場合はフォントサイズの指定が無視されます。normalを無視するルールは後述する他のフォントプロパティでも同じです。と、いうことは\verb+\fontsize+の中身を新たなclass名で置き換えれば、新しいフォントサイズを追加することが可能です。

例として、超巨大フォントという意味を込めて、\verb+\exHuge+を追加してみましょう。

\begin{code}{tex}
\newcommand{\exHuge}{\defstr\fontsize{exHugesize}}
\end{code}

{\color{red}\verb+\defstr+}は\Lamuriyanの処理系がコマンドの中身を文字列として読み取れるようにする為のコマンドです。\verb+\edef+の結果をString型として保持すると思ってください。このコマンドを使わないと\Lamuriyan処理系は内容を読み取ることが出来ません。

\section{フォントファミリを変更する}
フォントファミリを変更するには\verb+\mcfamily,\gtfamily,\rmfamily,\sffamily,\ttfamily+を使います。順に明朝体、ゴシック体、ローマン体、サンセリフ体、タイプライタ体をあらわあします。class名はコマンド名と全く同じです。

引数部分だけフォントファミリを変更する\verb+\textmc,\textrm,\textsf,\texttt+等も定義してあります。

\subsection{フォントファミリを追加する}
\ref[,章]{sec:addfontsize}で述べたのと同様にフォントファミリも追加することが出来ます。フォントファミリのclass名を定義しているコマンドは\verb+\fontfamily+です。


\section{太文字にする}
\verb+\bf+を使います。class名は"boldweight"です。引数の部分を太文字にする\verb+\textbf+もあります。

基本的にHTMLでは太文字にするか、普通の太さにするかしか使えないので太文字に関する新しい何かの追加をすることは無いと思いますが、これまで同様、\verb+\fontseries+の内容を変更することで太文字の定義を変えられます。

\section{斜体にする}
\verb+\itshape,\slshape+を使います。前者はイタリック体、後者は斜体を表します。日本語ではどっちも傾けるだけなので違いは無いらしいですが、英文では違いがあるそうです。\verb+\itshape+を指定するとclass名は"itstyle"が、\verb+\slshape+を指定するとclass名は"slstyle"になります。

斜体に関して新しい定義をしたい場合はこれまでと同様、\verb+\fontshape+の内容を変更します。

\section{スモールキャップ体にする}
\verb+\scshape+を使います。class名はscvariantです。

\TeXではスモールキャップ体もフォントシェイプの一部ですが、CSSでは別項目になので、それにそって\Lamuriyanでも別項目になっています。

これまで同様、新たに追加したい場合は、\verb+fontvariant+の内容を変更します。

\section{文字の色を設定する}
\verb+\color+を使います。色に関してはclass名で指定するのでは無く、直接"style=color:色"を書き出します。引数部分だけ色を指定する\verb+\textcolor+もあります。

色の指定にはいくつかの方法があります。

-\verb+\color{\#ffffff},\color{*ffffff}+の様に色を書く方法。これはそのまま"color:\#ffffff"となります。\verb+\#+と書くのは面倒くさいので、*を代用することが出来ます。
-\verb+\color{red}+の様に色の名前を書く方法。これもそのまま"color:red"となります。
-\verb+\color{rgba(255,255,255,1)}+の様に書く方法。これもそのまま"color:rgba(255,255,255,1)"となります。
-\verb+\color[\#rgb]{ffffff},\color[*rgb]{ffffff}+と書く方法。これは第二引数の内容に\#を付加して色の値にします。
-\verb+\color[rgb]{255,255,255}+と書く方法。これはrgbの各値を,区切りで0~255の値で表現します。
-\verb+\color[argb]{0.5,255,255,255},\color[argb]{127,255,255,255}+と書く方法。これはrgba(255,255,255,0.5)に変換されます。0.5の様に小数点を含む数でaを表現すると0~1の範囲で指定された物と見なし、整数で表現すると0~255で表現した物と見なします。{\color{\cred}ただし、1だけは0~1の範囲で指定されたと見なします}
-\verb+\color[a\#rgb]{0.5\#fff},\color[a*rgb]{0.5*fff}+の様に指定する方法。これはrgbに関しては\#FFFの形式で表し、aの値だけ別に0~1の範囲、もしくは0~255の範囲で指定します。\#,*の記号は一致させてください。なお、どちらも\verb+\color[a*rgb]{127,FFF}+のように「,」で代用することも可能です。


\section{文字に関するClassのまとめ}
\begin{tabular}{|C|L|}\hline
  \th class名&\th 説明\\\hline
  tinysize&一般に50\%サイズのフォント\\\hline
  scriptsize&一般に70\%サイズのフォント\\\hline
  footnotesize&一般に80\%サイズのフォント\\\hline
  largesize&一般に120\%サイズのフォント\\\hline
  xlargesize&一般に140\%サイズのフォント\\\hline
  xxlargesize&一般に170\%サイズのフォント\\\hline
  hugesize&一般に210\%サイズのフォント\\\hline
  xhugesize&一般に250\%サイズのフォント\\\hline
  mcfamily&明朝体フォント\\\hline
  gtfamily&ゴシック体フォント\\\hline
  rmfamily&ローマン体フォント\\\hline
  sffamily&サンセリフ体フォント\\\hline
  ttfamily&タイプライタ体フォント\\\hline
  boldweight&太文字\\\hline
  slstyle&斜体\\\hline
  itstyle&イタリック体\\\hline
  scvariant&スモールキャップ体\\\hline
\end{tabular}

\input{aboutDOM.tex}
\chapter{マクロ}
\Lamuriyanは当然マクロが使えます。が、nodamushiが\TeXの知識が曖昧すぎる為、動けば良いじゃんの精神にしたがってかなりテキトーな実装になっています。実際の\TeXはこうだよ、こういう動作にした方が良いよと言うことがありましたら、遠慮無く@nodamushiにリプを送って下さい。参考にします。参考にするだけで、実装するかは分かりません。

\chapternav

\section{マクロ呼び出し}
通常のマクロ呼び出しと同じように呼び出すのですが、LaTeX書いてて常々思うんですが、{\color{cred}\bfマクロ名に日本語使うことなくね?}私、マクロ名書いた後に日本語が来ると、半角/全角キーを押すことに集中してしまうが為、スペースキーを押すことを忘れ、エラーをよく出してます。むかつきます。

なので、\LamuriyanはCharCategoryのアルファベットは、a-z,A-Zのみを登録しています。日本語は使えません。よって「\verb+\LamuriyanはCharCategory…+」の様にマクロ名の後にスペース無く日本語を入力しても問題ありません。もちろん、マクロ名の後のスペースは無視されるので、書いても問題ありません。


\section{マクロを定義する}
\verb+\defと\newcommand+を定義しています。使い方は大体本家\TeXやLaTeXと同じだと思います。パラグラフの扱いがどうのこうのとかいうのをどっかで読んだんですが、正直よーわからんので、無視してます。改行どこまでも続いてもオーケー。

\section{展開したトークンを保持する}
\tdef\vb{\verb+\edefや\protected@edefについて+}
\subsection{\vb}
いわゆる\verb+\edef+に関しては、\verb+\edefと\protedted@edef+が定義されています。たぶん、どちらも大体LaTeXと同じ挙動を示すんじゃないかなーと思っています。

注意して貰いたい事は、\verb+\edef,\protedted@edef+はトークンを保持するだけであり、DOM(TeXで言えばBoxにあたるものか)を保持することは出来ません。\verb+\edef,\protected@edef+内部でDOMを作成した場合、環境に挿入されます。\verb+\Lamuriyan+(=\Lamuriyan)などはDOMを生成するコマンドなので、うっかり混ぜてしまうことがないよう注意してください。

\tdef\vb{引数でないブロックを扱う\verb+\tdefについて+}
\subsection{\vb}
\Lamuriyanには\verb+\tdef+という特殊なトークンを展開し、保持するマクロが定義されています。基本的な使い方は\yen edefとほぼ同じです。

\verb+\edef+の欠点として、\yen verbを使うことが出来ないことがあります。これは、\verb+\edefの{~}が引数であり、{}の中身がトークンとして先に読み込まれてしまい、\verbを実行できない為です。この欠点を補う為に、\tdefを定義しておきました。+

\verb+\tdefにおいて、{~}は+{\color{red} 引数ではありません}。\verb+\tdefにおける引数は\tdef#1となっています。#1は\nameの様なエスケープシーケンスです。\tdefが発行された後には必ず、一番最初に「{」が来なくてはなりません。なお、これは展開された結果一番最初に「{」があれば良く、以下の様に\hogeが展開され、\bgroupが展開された結果「{」が来るような場合は許可されます。また、\tdefによる定義中にで\tdefを使うことは出来ません。これらのルールを破った場合は\tdefは定義を行いません。+

\begin{code}{tex}
\def\hoge{\bgroup \verb}
\tdef\hoge\hoge+\(^_^)\+}
\end{code}

さらに、上の例では\verb+\tdefで\hogeの定義を書き換えていますが、実際に\tdefが定義を行うのは}の後であるため、\hogeの後の二つ目の\hogeは\bgroup \verbに展開されます。\hogeの実行例↓+

{
\def\hoge{\bgroup \verb}
\tdef\hoge\hoge+\(^_^)\+}
\hoge
}

また、\verb+\edef+等と同じく、\verb+\globalを付加することが出来ます。+

なお、\verb+\edef同様、\noexpand,\unexpandedを使うことが出来ますが、\protected@tdefは定義されていません。理由は{~}が引数でないが故、\protectを限定的に再定義することが出来なかったからです。それでも\protectを使いたい場合は下のように自分で定義してください。+
\begin{code}{tex}
\def\protect{\noexpand\protect\noexpand}
  % ちなみに、以下の様にしても良いです。
  % \makeatletter \let\protect\@unexpandable@protect \makeatother
\tdef\name{~~}
\let\protect\relax
\end{code}
\section{リスト}
配列構造を扱う為にわざわざ,区切りの文字列で扱うとか、マジ意味わかんない。なんなの?苦行なの?マゾなの?私嫌よ。

というわけで、\Lamuriyanはリストが使えます。

\subsection{リスト生成}
「\verb+\newlist リスト名+」で生成します。

\begin{code}{tex}
  \newlist \listname
\end{code}
の様に宣言すると、\verb+\listname+というリストを生成することが出来ます。


\subsection{追加関連(set push unshift)}
「\verb+\set リスト名 数字 追加する内容+」でリストの数字番目の要素を設定します。

「\verb+\push リスト名 追加する内容+」でリストの最後に内容を追加します。

「\verb+\unshift リスト名 追加する内容+」でリストの最初に内容を追加します。

\begin{code}{tex}
\set\listname 5 {\Lamuriyanです}
\push\listname{\Lamuriyanです}
\unshift\listname{\Lamuriyanです}
\end{code}


\subsection{取得関連(get pop shift)}
「\verb+\get リスト名 数字+」でリストの数字番目の要素を取得します。

「\verb+\pop リスト名+」でリストの最後の要素を取得し、削除します。

「\verb+\shift リスト名+」でリストの最初の要素を取得し、削除します。


\begin{code}{tex}
\get \listname 5
\pop\listname
\shift\listname
\end{code}




\subsection{長さ}
「\verb+\length リスト名+」でリストの長さを返します。リストでない場合は-1が返ります。

\subsection{リストが空かどうか}
「\verb+\ifemptylist リスト名 空の場合 \else 空で無いの場合 \fi+」というif構文が使えます。


\subsection{各要素に対して処理をする}
foreach構文を用意しています。
\begin{code}{tex}
\@foreach \@memory:=\listname\do{"\@memory"}
\end{code}
と書くと、\verb+\listnameの内容をすべて""で囲って出力します。+

例として、次のコードを\Lamuriyanソースに書いてみます。
\begin{code}{tex}
{%makeatletterや\testlistの有効範囲を限定しておく為
\makeatletter %  @をマクロ名として認識する為です
\newlist\testlist % \testlistの作成
\push\testlist{りんご}
\push\testlist{ごりら}
\push\testlist{らいお\textcolor{red}{ん}}
出力結果「\@foreach \@memory:=\testlist\do{"\@memory"}」
}
\end{code}
{
\makeatletter
\newlist\testlist
\push\testlist{りんご}
\push\testlist{ごりら}
\push\testlist{らいお\textcolor{red}{ん}}
出力結果「\@foreach \@memory:=\testlist\do{"\@memory"}」
}
\section{今後の予定}
なんだかんだいっても、やっぱ複雑な\TeXマクロ書くの面倒くさいにゃ!というわけで、マクロの内容をGroovyでかける\verb+\directgroovy+を作ろうかと思ってるんだけど、どういう風に定義し、実行するのかfixしてないので絶賛放置中。


\chapter{数式}
\TeXで文章を書く最大のメリットは数式を簡単に書けるということでしょう。\nTeXでは数式をMathMLに変換して出力します。

ブラウザでMathMLを表示するには以下のどちらかを利用してください。
+MathML対応ブラウザで表示
+MathJax利用する

現在ではMathMLへのブラウザの対応状況は悪く、MathJaxを利用するのが無難かと思われます。


\chapternav

\section{数式の書き方}

\nTeXでは数式は\verb+$数式$,$$数式$$,\[数式\]+,eqnarray環境,eqnarray*環境を用いることで書くことが出来ます。

\subsection{\$数式\$形式}
\varb+$数式$+は「$x=y$」の様にインライン形式として出力されます。ここでは大型演算子$\sum$などは$\sum_0^1$の様になります。

インラインとブロック形式のHTMLの違いとしては、<math>にdisplay="block"属性が付くか付かないかの違いしかありませんが。

\subsection{\$\$数式\$\$形式}
\verb+$$数式$$+はブロック形式の数式として出力されます。$$x+y$$の様になります。
\verb+\[と\]は+\nTeXではどちらも\verb+$$に置換されるだけなので、実際には\[数式\[の様に書いても数式と判断されます。これ豆な(-_-)+

ここでは$\sum$などの大型演算子は$$\sum_0^1$$の様に上下につきます。

\subsection{eqnarray環境}
\TeXと同じように、数式に番号を付けることが出来、数式の位置揃えが出来る環境です。

\begin{code}{tex}
\begin{eqnarray}
  \label{eq:微分の定義}
  \frac{df(x)}{dx} &=&\lim_{\delta x\to0}\frac{f(x+\delta x)-f(x)}{\delta x}\\
  \label{eq:偏微分の定義}
  \frac{\partial f(x,y,z...)}{\partial x}&=&
  \lim_{\delta x\to0}\frac{f(x+\delta x,y,z,...)-f(x,y,z,...)}{\delta x}
\end{eqnarray}
\end{code}
\begin{eqnarray}
  \label{eq:微分の定義}
  \frac{df(x)}{dx} &=&\lim_{\delta x\to0}\frac{f(x+\delta x)-f(x)}{\delta x}\\
  \label{eq:偏微分の定義}
  \frac{\partial f(x,y,z...)}{\partial x}&=&
  \lim_{\delta x\to0}\frac{f(x+\delta x,y,z,...)-f(x,y,z,...)}{\delta x}
\end{eqnarray}

一つ目の要素は右寄せ、二つ目は中央寄せ、三つ目は左寄せになります。本家\TeXと違って、\nTeXの表はいくらでも拡張可能なので、三つ以上&があっても、問題なく表示されます。その際、位置揃えは(たぶん)中央寄せになります。

デフォルトでは数式の番号は(チャプター番号:セクション番号-セクション内での式の番号)になります。チャプター番号とセクション番号が0の時はそれぞれ省略されます。これらの設定を変更したい場合は\verb+\theeqnumber+を書き換えてください。全部の行で式番号が要らない場合はeqnarray*環境を用いてください。

\verb+\nonumber+を用いることで、その行の数式は書かれないようになります。
\begin{eqnarray}
  f(x)&=&x^{100}+x^{99}+x^{98}+x^{96}+..............\nonumber\\
  &&...................+x^3+x^2+x+1
\end{eqnarray}

\section{array環境}
行列などの表組の数式を書きたい場合に使います。\TeXと同様\verb+\begin{array}{crl}+の様に使います

\begin{code}{tex}
$$
A = \left(
  \begin{array}{ccc}
    1 & 0 & 0\\
    0 & 1 & 0\\
    0 & 0& 1
  \end{array}\right)
$$
\end{code}

$$
A = \left(
  \begin{array}{ccc}
    1 & 0 & 0\\
    0 & 1 & 0\\
    0 & 0& 1
  \end{array}\right)
$$

\subsection{array環境における縦線横線について}

\verb+\hlineや環境の引数の|+には現在対応していません。というのも、MathMLの仕様についてよく理解してないからです。( ^ω^)ゴメンネ

\section{MathJaxを使う問題点}

\verb+\label+を宣言すると、該当ノードにIDを設定し\verb+\ref+ではそのIDにリンクを張る仕様になっていますが、MathJaxを使うと、数式中で定義したIDが消えてしまい、リンク先がなくなってしまう問題があります。

通常の\verb+\ref+を用いて書くと、リンクが生成されてしまうので、\verb+\nolinkref+を用いるか、convsetting.hcvのconvert項目のRefLabelConverterのオプションにmlabeledtrを追加してください。\verb+\nolinkref+を用いるとこれ単体がリンクを使わない参照になります。後者のコンバーターの設定を用いると、\verb+\ref+を用いても、\verb+\ref+が付くノードの名前が引数に一致した場合は、リンクを使わない参照を作るようにします。デフォルトではこの設定が付いていますので、数式への参照はリンクが生成されません。MathJaxは使わないので、数式へのリンクが生成されて欲しい場合は、\\[1.5em]
reflabel>nodamushi.ntex.html.RefLabelConverter:mlabeledtr,math\\[1.5em]
という設定の後ろにある「mlabeledtr,math」を削除してください。

\section{数式モードで使える実体参照}
ひらく様の\a{http://www.hinet.mydns.jp/?TeX_mathml.rb}{ひらくの工房-TeX_mathml.rb}に載っている記号については、参考という名の丸コピいたしました。これ書いてるときにページ見てて気がついたのですが、別ページにまだいっぱい作ってない記号が載っていました。ぶっちゃけ、これ作った本人が$\sum \int \lim$ぐらいがあれば何も問題ないというゴミ屑なので、面倒くさいので放置です( ^ω^)

\subsection{足りない実体参照を自作する}

あの実体参照がないとか、人を馬鹿にしてんじゃねーぞ死ねnodamushi!!という事が多分にあるでしょう。その意見にはおおむね賛成ですが、何も貴方が犯罪者になる必要はありません。自分で定義しましょう。

実体参照もマクロを使って定義しています。そのためのマクロが\verb+\mathescape+です。

\verb+\mathescape#1#2,#3+という構文で、#1は登録するコマンドシーケンス名を、#2は実体参照の文字列を、「,」を挟んで#3は<mo><mi><mn><mtext>のどれでくくるかの指定をします。指定には以下の一文字を使います。

\begin{tabular}{|C|l|}\hline
  \th第三引数&\th 説明\\\hline
  o,O&<mo>で囲みます。+や-や(などの演算子に付けます\\\hline
  n,N&<mn>で囲みます。数値を表します\\\hline
  t,T&<mtext>で囲みます。文字列を表します。\\\hline
  i,I&<mi>で囲います。xとかyとか基本的に何でも<mi>には入るみたいです。\\\hline
  a,A&<mi>で囲います。i,Iと違い、mathvariant=normalの属性がつきます。

なんでa,Aかって?nodamushiの語学力のなさが原因だよ。何も思いつかなかったからアルファベットの最初でいいやってだけだよ(-_-)
  \\\hline
\end{tabular}

大文字と小文字の違いは、大文字の場合$$\sum_0^10$$の様に上下に付くようになります。\\[5em]

では、実際に定義をしてみましょう。実体参照には&と#の文字があって、そのままだと毎回\verb+\&,\#+と書かないといけません。これは面倒ですのでこれらのカテゴリコードを通常の文字列に変更してしまいましょう。他の範囲に影響を出さない為にグループで囲っておきます。

\begin{code}{tex}
\bgroup
\catcode `\&=12
\catcode `\#=12

\global\mathescape\sum &sum;,O

\egroup
\end{code}



\section{関数、演算子を定義する}

\verb+\mathescape+でも定義できますが、一箇所使うだけだってのに、いちいち定義なんてしてらんない、って時には\verb+\operatorname+を使いましょう。これは\TeXと動作は一緒で、関数を定義するときに使います。
\begin{code}{tex}
\[
{arg\,max}_{x\in X}\text{と}\operatorname*{arg\,max}_{x\in X} sin^2\text{と}\operatorname{sin}^2
\]
\end{code}
\[
{arg\,max}_{x\in X}\text{と}\operatorname*{arg\,max}_{x\in X} sin^2\text{と}\operatorname{sin}^2
\]

演算子の場合は\verb+\mathbin,\mathop+が使えます。

\section{マクロに関して}
\verb+\over,\under,\frac,\sqrt,\text+など、基本的な物は最初から実装してあります。

\subsection{$\sideset{_n}{_m}P$とかの前後に付く記号を書きたい}
\verb+\sideset+を使うことが出来ます。
\begin{code}{tex}
\[
\sideset{_n}{_m}{P}
\]
\end{code}
\[
\sideset{_n}{_m}{P}
\]


\subsection{<mmultiscripts>を書きたい}
\verb+\multiscripts+を使うことが出来ます。このマクロは
\begin{verbatim}
#4  #5
  #1
#2  #3
\end{verbatim}
の様な配置になります。何も表記したくない場所は\verb+\noneNode+と記入してください。
\begin{code}{tex}
\[
\multiscripts{M}\alpha\noneNode{\beta+1}\omega
\]
\end{code}
\[
\multiscripts{M}\alpha\noneNode{\beta+1}\omega
\]

\subsection{<menclose>の角丸や円を書きたい}
\verb+\roundedbox,\circle+が使えます。書式は\verb+\roundedbox[#1]#2+です。第一引数は<menclose>の属性を,区切りで並べます。(id=rbox,class=rboxclassの様に)。第二引数は<menclose>の中身です。
\begin{code}{tex}
\[
\roundedbox{x} \circle{y}
\]
\end{code}
\[
\roundedbox{x} \circle{y}
\]


\subsection{数式に打ち消し線を書きたい}
\verb+\cancel+が使えます。書式は\verb+\cancel[#1]#2+です。第一引数はu,v,h,dのうちいずれか一文字を選択してください。省略した場合はuになります。実装は<menclose>の~strikeを使っています。第一引数も、これの一文字目です。

\begin{code}{tex}
\[
  \frac{1}{\cancel{(a+b)}}\frac{\cancel[d]{(a+b)}}{\cancel[h]{x}}\cancel[v]{x}
\]
\end{code}
\[
  \frac{1}{\cancel{(a+b)}}\frac{\cancel[d]{(a+b)}}{\cancel[h]{x}}\cancel[v]{x}
\]

\subsection{ていうか、<menclose>書きたい}
\verb+\enclose+が使えます。書式は\verb+\enclose[#1]#2#3+です。第一引数は<menclose>の属性を,区切りで並べます。第二引数は<menclose>のnotation属性を指定します。第三引数は<menclose>の中身です。
\begin{code}{tex}
\[
\enclose{actuarial}{a^2+b^2}
\]
\end{code}
\[
\enclose{actuarial}{a^2+b^2}
\]


\section{自作マクロについて}
ここに書いてない機能については、自作マクロを作成するしかありません。

数式モードでマクロは基本的には\verb+\blockと\mfragと\setAttr,\setAttrs+を駆使してDOMを生成していくことで作ります。

これらは\nTeXにおける他のマクロの作り方(今度書く)と同様ですが、いくつか特殊なルールがあります。

\subsection{<mrow>で囲うべきか判断できない場合}
\label{sec:mfrag}
MathMLにおいて、<mrow>は様々な場面で使われる。例えば、<msup>などは二つの要素を持つが、中に数式を入れたい場合は\\[3em]
<msup><mrow><mo>(</mo><mi>x</mi><mo>+</mo><mn>1</mn><mo>)</mo></mrow> <mn>2</mn></msup>\\[3em]
といった具合に<mrow>で囲う必要がある。しかし、後ろの<mn>2</mn>は要素が一つしかないので、<mrow>で囲う必要性はない。むろん、囲っても構わないのだが、\textbf{美しくない}。要素が一つなら<mrow>で囲わないのがジャスティスである。しかし、これからいくつの要素が入るかなんてマクロ実行段階で知る術はない。

そこで、\nTeXでは数式モードにおいて特殊なタグ<mfrag>というタグを定義している。このタグは要素が一つしかない場合はDocumentFragmentとして働き、要素が二つ以上あるときは<mrow>になるというタグだ。\verb+\mfrag#1+マクロで<mfrag>を生成できる。後は#1の部分に中身を書けば良い。

\subsection{スコープの為のグループは作らない}
\nTeXでは数式モードにおいて、begingroupフックを利用している。グループの中身を\ref[,章]{sec:mfrag}にて説明した<mfrag>で囲う命令を発効する為である。こうしなければ、\verb|{x+1}^2|が<mi>x</mi><mo>+</mo><msup><mn>1</mn><mn>2</mn></msup>となってしまうからである。

そこで、\nTeXでは\{でグループが作られるたびに(マクロの引数のグループは関係ない)設定したトークンをアウトプットするeverybegingroupというフックを用意し、これを解決した。しかし、これが逆にマクロを作る上で問題となることがある。

例えば、引数をそのまま\{#1\}と書き出した場合、この\{に反応して\mfragが生成される。それは意図した挙動でないことが多いだろう。したがって、数式モードで使うマクロでは\{\}を用いてグループを作らないことが賢明である。

それでもグループが必要な場合には、\verb+\bgroup,\egroup+のペアを用いる。数式モードでは\verb+\bgroup,\egroup+はeverybegingroupのフックが呼び出されない。

\subsection{ある範囲を一つの<mo><mi><mn><mtext>にしたい}
\nTeXは数式モードにおいて、\verb+\@startchargroup+コマンドが発行されてから、\verb+\@endchargroup+が現れるまでの間を一つの<mo><mi><mn><mtext>と認識します。\verb+\@startchargroup+の引数は「o,i,n,t」のいずれか一文字です。それぞれ、<mo><mi><mn><mtext>に対応しています。\verb+\mathopや\operator+などはこれを利用しています。

\subsection{^_を上下につけるか横に付けるかを変更する}
あるノードに対して^_の位置はそのノードに対して\verb+\@underoverflag+が呼ばれているかどうかで決まる。
\begin{code}{tex}
\block{nodename}{\@underoverflag 内容}
\end{code}
の様に書くとこの<nodename>は上下に^_が付く。





%\end{div}
\end{document}