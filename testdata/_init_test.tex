%このファイルを編集することでファイルを読み込む前のパーサーの状態を
%設定することが出来ます。nodamushiの変なお遊びとか、
%迷走している様子が大量に残っています

%\indexitem depth number content type
\newcommand{\indexitem}[4]{%
  \bgroup%
  \disableparagraph%
%  \settingmarkg\@indexitem@memory%
   \settingmark\@indexitem@memory%
   \bgroup
   \block{ignore}{#3}%
   \egroup%
  \lastsiblingmark\@indexitem@lastsibmemory%
  \@indexitem\@indexitem@memory{#1}{#2}\@indexitem@lastsibmemory{#4}%
%  \settingnodeg\@indexitem@memory%
  \settingnode\@indexitem@memory%
  \egroup%
}
\def\navtitle{目次}
% \nav type depth scope     \nav{chapter,section}{1-3}{allまたはhere}の様に記述する。
% typeは目次に表示するindexitemのtypeの種類を,区切りで指定する。上の例ではchapterとsectionを指定している。
% depthは表示するchapterやsectionのdepthを表す。上の例ではdepth=1から3までを集めた目次を作成する。
% chapterだけの目次を作りたい場合は0を、chpaterとsectionまでだったら0-1の様に設定する。
% scopeはallの場合は、上の例では登録されているchapter,section全てのナビを作成する。
% hereの場合、例えばchapterの次にnavが来た場合は、そのchapterの中に入るsectionだけを目次にする。
% 同様にsectionの中でnavを作った場合は、そのsectionに入るsectionだけを目次にする。
\newcommand{\nav}[3]{%
  \createNode[b]{convertnav}%
  \setAttr{type}{#1}%
  \setAttr{depth}{#2}%
  \setAttr{scope}{#3}%
  \setAttr{title}{\navtitle}%
  \indexitem 000{nav}}
\newcommand{\chapternav}[1][1-3]{%
  \nav{chapter,section}{#1}{here}}
\newcommand{\documentnav}[1][0-1]{%
  \nav{chapter,section}{#1}{all}}
%\newcounter{catcodeimpcounter}%catcodeで使う
\newlist\cssfiles
\newcommand{\css}[2][]{%
  \bgroup%
  \let\oldparagraph\useparagraph%
  \disableparagraph%
  \block[rel=stylesheet,type=text/css]{link}{\setAttr{href}{#2}\setAttrs{#1}}%
  \push\cssfiles{#1}%
  \resetparagraph\oldparagraph%
  \egroup}
\newcommand{\charset}[1]{
  \setgprop{charset}{#1}%
  \block{meta}{\setAttr{content}{text/html;charset=#1}\setAttr{http-equiv}{Content-Type}}}
  

\newlist\jsfiles%
\newcommand{\js}[2][]{%
  \bgroup%
  \let\oldparagraph\useparagraph%
  \disableparagraph%
  \block[type=text/javascript]{script}{\setAttrs{#1}\setAttr{src}{#2}  }%
  \resetparagraph\oldparagraph%
  \egroup}

\def\articletitle{}
\def\title#1{\defstr\articletitle{#1}}


%%%% 環境定義%%%%%
\newenvironment{div}{\tagname{div}}{}


%環境のignoreプロパティは値がtrueの時、子要素を含めて
%HTMLにコンバートしない。デフォルトではこの値はfalseである。
\def\ignore{\setprop{ignore}{true}}


\newverbenvironment{script}[1][type=text/javascript]{%
  \tagname{script}%
  \setAttrs{#1}%
  \disableparagraph}{}

%<header><h1>#1</h1></header>をつくる
\def\createheader#1{%
  \disableparagraph%
  \block{header}{\block{h1}{#1}}%
  \enableparagraph}

\newcommand{\blockmarker}[2][]{\createNode[b]{block-#2}\setAttrs{#1}}

\newverbenvironment{htmlcode}{%
  \tagname{htmlcode}%
  \notescapechar%
  \disableparagraph}{}
\newcommand{\inputhtml}[1]{%
  \begin{text}\tagname{htmlcode}%
    \disableparagraph\notescapechar%
    \verbinput{#1}%
  \end{text}}

\newcommand{\a}[3][]{%
  \inline[#1]{a}{\setAttr{href}{#2}#3}}

\def\abstracttitle{概要}
\newenvironment{abstract}{\tagname{section}%
  \labelable{\abstracttitle}%
  \setAttr{class}{abstract}%
  \createheader\abstracttitle}{}
\newif{\ifnowchapter}


% 
% デフォルトではnodamushiのホームページEGGw用に
% chapterは別ページのarticleに変換されることを前提とした設定になっています
% 
\newcommand{\chapter}[2][]{%
  \stepcounter{chapter}%
  \nowchaptertrue%
  \blockmarker{chapter}%
  \setAttr{number}{第\thechapter章}%
  \setAttr{title}{#2}%
  \setAttr{idhash}{section}%
  \ifempty{#1}\else%
  \createIDfromeHashKey{#1}%
  \fi%
  \labelable{\thechapter}%
  \indexitem 0\thechapter{#2}{chapter}%
}
  

\newcommand{\@makesectitles}[2]{%
  \inline[class=secnumber]{span}{#1}{#2}}
\newcommand{\makesectiontitle}[1]{%
  \@makesectitles{\thesection章}{#1}}

\newcommand{\section}[2][]{%
  \stepcounter{section}%
  \blockmarker[class=sect]{section}%
  \setAttr{idhash}{section}%
  \ifempty{#1}\else%
  \createIDfromeHashKey{#1}%
  \fi%
  \labelable{\thesection}%
  \indexitem 1\thesection{#2}{section}%
  \bgroup%
  \settingmark\impmark@sect%
  \createheader{\makesectiontitle{#2}}%
  \settingnode\impmark@sect\egroup}

\newcommand{\makesubsectiontitle}[1]{%
  \@makesectitles{\thesubsection}{#1}}
\newcommand{\subsection}[2][]{%
  \stepcounter{subsection}%
  \blockmarker[class=ssect]{subsection}%
  \setAttr{idhash}{section}%
  \ifempty{#1}\else%
  \createIDfromeHashKey{#1}%
  \fi%
  \labelable{\thesubsection}%
  \indexitem 2\thesubsection{#2}{section}%
  \bgroup%
  \settingmark\impmark@sect%
  \createheader{\makesubsectiontitle{#2}}%
  \settingnode\impmark@sect\egroup}

\newcommand{\makesubsubsectiontitle}[1]{%
  \@makesectitles\thesubsubsection{#1}}
\newcommand{\subsubsection}[2][]{%
  \stepcounter{subsubsection}%
  \blockmarker[class=sssect]{subsubsection}%
  \setAttr{idhash}{section}%
  \ifempty{#1}\else%
  \createIDfromeHashKey{#1}%
  \fi%
  \labelable{\thesubsubsection}%
  \indexitem 3\thesubsubsection{#2}{section}%
  \bgroup%
  \settingmark\impmark@sect%
  \createheader{\makesubsubsectiontitle{#2}}%
  \settingnode\impmark@sect\egroup}

\newcommand{\makesubsubsubsectiontitle}[1]{%
  \@makesectitles\thesubsubsubsection{#1}}
\newcommand{\subsubsubsection}[2][]{%
  \stepcounter{subsubsubsection}%
  \blockmarker[class=ssssect]{subsubsubsection}%
  \setAttr{idhash}{section}%
  \ifempty{#1}\else%
  \createIDfromeHashKey{#1}%
  \fi%
  \labelable{\thesubsubsubsection}%
  \indexitem 4\thesubsubsubsection{#2}{section}%
  \bgroup%
  \settingmark\impmark@sect%
  \createheader{\makesubsubsubsectiontitle{#2}}%
  \settingnode\impmark@sect\egroup}

\def\chapterend{\blockmarker{end-chapter}\nowchapterfalse\setcounter{section}{-1}\stepcounter{section}}
\def\sectionend{\blockmarker{end-section}}
\def\subsectionend{\blockmarker{end-subsection}}
\def\subsubsectionend{\blockmarker{end-subsubsection}}
\def\subsubsubsectionend{\blockmarker{end-subsubsubsection}}

%<b>
\newcommand{\b}[2][]{\inline[#1]{b}{#2}}
%<a>
%基本は\anchor
%一応\aと\linkも定義
\newcommand{\anchor}[3][]{\inline[#1]{a}{\setAttr{href}{#2}#3}}
\let\a\anchor
\let\link\anchor

% 引用環境
% <cite>の始まりの文字列、終わりの文字列を変更したい場合は
% \prequotecite,\postquoteciteを変更する
\def\prequotecite{引用:}
\def\postquotecite{}

\def\@blockquote@makeurl#1{%
  \prequotecite%
  \ifx\@blockquote@url\undefined #1\else%
  \anchor{\@blockquote@url}{#1}%
%  \inline[href=\@blockquote@url]{a}{#1}\fi%
  \postquotecite}

\def\@blockquote@makecite{%
  \ifx \@blockquote@title\undefined \else%
  \let\oldparagraph\useparagraph%
  \disableparagraph%
  \block{cite}{\@blockquote@makeurl\@blockquote@title}%
  \resetparagraph\oldparagraph%  
  \fi}

\def\@blockquote@citeattr{%
  \ifx\@blockquote@url\undefined \else%
  \seteAttr{cite}{\@blockquote@url}%
  \fi}

\def\@blockquote@makecs#1#2{%
  \expandafter\def\csname @blockquote@#1\endcsname{#2}}

% \begin{quotatino}[url=http//~~~~,title=~~~~,position=tかb]
%   positionは<cite>を最初に入れるか最後に入れるかの指定。
%   tはtop、bはbottom デフォルトはt
\newenvironment{quotation}[1][]{%
  \let\@blockquote@url\undefined%
  \let\@blockquote@title\undefined%
  \def\@blockquote@position{t}%
  \tagname{blockquote}%
  \@for\@blockquote@temp:=#1\do{%
    \expandafter\@divide@eq\@blockquote@temp\@vid\@blockquote@makecs}%
  \@blockquote@citeattr%
  \expandafter\ifx\@blockquote@position t \@blockquote@makecite\fi%
  \bgroup% 
}{\egroup \expandafter\ifx\@blockquote@position b \@blockquote@makecite\fi}


%簡易記法の設定
% ファイルの改行フックと、謎の行頭のみアクティブキャラクターにするマクロを使って、よくある
% 行頭に+や-が来るとリストに変換する機能を実装
% はてな記法とかよくあるのと同じように、+はenumerateに、-はitemizeに変換される。
% +++とか複数付けると、それだけ階層を作れるのが一般的だけど、ちょっとそれを実装するのは面倒くさい…
% 数を数えるところまでは出来るんだけどねぇ。コンバーターに頼らないと無理かと。
\newcommand{\slnewlineA}[1]{\everynewline={\slnewlineB{#1}}}
\newcommand{\slnewlineB}[1]{%
  \end{#1}}
\newcounter{slplus}
\newcounter{slminus}
\setcounter{slplus}{-100}
\setcounter{slminus}{-100}
\newcommand{\@slplus}{%
  \stepcounter{slplus}\@simplelist{slplus}{enumerate}}
\newcommand{\@slminus}{%
  \stepcounter{slminus}\@simplelist{slminus}{itemize}}
\newcommand{\@simplelist}[2]{%
  \ifnum\value{#1}=\fileline \relax \else%
  \setcounter{#1}{\fileline}\begin{#2}\onNLhook\fi%
  \item\everynewline={\slnewlineA{#2}}}
% \startlineactivecharは次の文字列はキャラクター、次はエスケープシーケンスです。
\startlineactivechar +\@slplus
\startlineactivechar -\@slminus

% 簡易引用記法 行頭の>>から行頭の<<までをquotation環境に変換する
% LaTeXだと、!と?の逆が出力されるみたいだけど、!`や?`を使えばいいし、いいよね?

\newcommand{\@ltactive}[1]{%\elseにたどり着くまで怒濤の\expandafter祭り
  \ifx >#1 \expandafter\begin\expandafter{%
      \expandafter q\expandafter u\expandafter o\expandafter t%
      \expandafter a\expandafter t\expandafter i\expandafter o\expandafter n%
    \expandafter}\else>#1\fi}
\newcommand{\@gtactive}[1]{%
  \ifx <#1 \end{quotation}\else <#1\fi}
\startlineactivechar >\@ltactive
\startlineactivechar <\@gtactive

\newcounter{figure}[section]
\renewcommand{\thefigure}{%
  \ifnum\value{chapter}>0 \arabic{chapter}:\fi%
  \ifnum\value{section}>0 \arabic{section}-\fi\arabic{figure}%
}

\newcommand{\@caption}[2]{
  \stepcounter{#1}%
  \block{figcaption}{%
    \expandafter\labelable\expandafter{\csname the#1\endcsname}%
    \@makecaption{#1}{\csname the#1\endcsname}{#2}}%
}

\newcommand{\@makecaption}[3]{%
  \csname capname@#1\endcsname#2\csname capseparator@#1\endcsname#3}


\def\@zoomstyle#1#2{zoom:#1;-moz-transform-origin:0 0;-moz-transform:scale(#1);}%#1:0.5など小数表記、#2:50%など%表記
\def\capname@figure{図}
\def\capseparator@figure{: }
\newenvironment{figure}{%
  \tagname{figcaption}%
  \disableparagraph%
  \setAttr{class}{figfigure}%
  \def\caption{\@caption{figure}}%
}{}
\def\linewidth{100\%}
\newcommand{\includegraphics}[2][]{%
  \inline[\@imagestyle{#1}{#2}]{img}{\setAttr{src}{#2}}%TODO ファイルの検索、rootdocumentに使ったファイルとして保存
}



\def\capname@table{表}
\def\capseparator@table{: }
\newcounter{table}[section]
\newenvironment{table}{%
  \tagname{figcaption}%
  \setAttr{class}{tabfigure}%
  \disableparagraph%
  \def\caption{\@caption{table}}%
}{}





