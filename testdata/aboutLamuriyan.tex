\chapter{Lamuriyanについて}
\label{sec:aboutlamuriyan}
\chapternav

\section{作り始めた動機}
\label{sec:motivation}
  私がいい加減ブログの記事を自分で読む時に探すのが大変なので、HPにまとめようかとしていたときに、やたら私の記事は数式が多いので\TeXでホームページ管理できないかなぁと考え、探してみましたが、
+Javaで使える
+DOM構造をはき出す
+relをリンクに変換する
+マクロが使える

ぐらいの条件で数秒探してみましたが、残念なことに私は探すことが出来ませんでした。

ある女王は言ったそうです。\textbf{\color{*5c9}パンがないなら作ればいいじゃない。}

私もその言葉に従うことにしました。ないなら作ればいいと。そして気がついたのです。\\[4em]

{\large あ、私、\TeXで引数とるマクロすら書いたことなかったわ}

\TeX仕様どころか、引数とるマクロすら書いたことがないのに処理環境を作り始めた世紀の愚者nodamushiによる\TeXライクな何か\Lamuriyanはこうして始まったのです。



\section{\Lamuriyanの方針}
\LamuriyanはHTMLのDOM構造を出力することを最終目標としており、{\color{red}デザインをすることやサイトを構築することを目的としていません。}

\Lamuriyanで出力された結果を別に作ってあるテンプレートに挿入してHTML全体を作る、例えば{\color{*f65}記事の内容を作るような場面}を想定して作っています。もちろん、やろうと思えば\Lamuriyanだけでデザインもすることは出来ますが、わざわざ\Lamuriyanで行うメリットが見当たりません。

デザインはCSSが担当すべきであり、HTMLが担当するべき事ではありません。そのため、\Lamuriyanはなるべくデザインに関する内容は{\color{red}classを出力することで表現します。}たとえば、太文字にしろという\verb+\bftext+などは、<span style="font-weight:bold">を出力するのでは無く、<span class="boldweight">という結果を出力します。実際に太文字にするかどうか、どういうデザインにするかはCSSに一任します。なので、\Lamuriyan上では太文字にしろ、という命令でも、実際には細字にしても構いませんし、テキストシャドウを用いて表現しても構いません。

サイト構築に関しても、\Lamuriyanの立ち位置はページを出力、もしくはページの一部を生成するツールという立ち位置を意識しており、サイト管理、構成は\Lamuriyanを用いるプログラムに任せることを前提としています。\\[3em]

\ref[,章]{sec:motivation}でも述べたように、処理系を作り始めるにはお粗末な\TeXの知識しか無く、\textbf{\color{*f46}\TeXとの完全な互換を目的とした物では\udecorationありません。}

あくまで、\TeXライクな\textbf{何か}であり、過去の資産の再利用は気にしていません。過去に書いた\TeXの記事を軽い修正でHTMLに出来ればな、ぐらいの心づもりです。

過去の資産の利用ではなく、これから記事を書いていくときに、\textbf{\color{*36f}\TeX形式の数式を使うことが出来}、\textbf{\color{*36f}数式などの番号管理をする必要が無く}、\textbf{\color{*36f}<section>タグとかも書かなくても自動で生成され}、\textbf{\color{*36f}同じ所はマクロで自動化することが出来}、あと私が過去に作った\textbf{\color{*36f}ソースコードをシンタックスハイライトするプログラムを自動で呼び出すことが出来る}、そんなシステムを目指しています。

その他の精神は以下のような物です。
+動けばいい
+実行速度は求めない
+省メモリは求めない
+本家\TeXと挙動が違う!バグだ!?否、仕様である

\vspace{2em}

\Lamuriyanを単純にMathMLを出力するツールとしてだけ使いたいという要望が万一ありましたら、DOMを生成するのでは無く、直接文字列を出力する為の簡易ツールを公開する………かもしれません。

\section{\Lamuriyanの由来}
\Lamuriyanは{\bf\color{red} n}odamushiが作る\TeX環境ということで、nTeXと呼んでいましたが、ググってみたら普通にいっぱい出てきました。というわけでnTeXから改名して\Lamuriyanという名前に変えました。この仮本部はnTeXの名前で作ったので、URLがntexです。

名前の由来は「LaTeX目指してみたけど、マジ無理やん」を略して「{\large La無理やん}」です。漢字ひらがなローマ字混じりだと打つのが面倒くさいので{\Lamuriyan}となりました。\Lamuriyanでは\verb+\Lamuriyan+とかくことで表示されます。頭文字のLAはLaTeXから来てるので、AがLの上に乗ります。\Lamuriyanと複雑な表記をしたくないときはLamuriyanと書いてください。





